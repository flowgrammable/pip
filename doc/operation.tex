
\section{Operational Semantics}

\subsection{Notation}

The state of the evaluator, $\sigma$, is the tuple consisting of \\
packet, $\pi$\\
action\_list, $\alpha$ \\
stored\_action\_list, $\Lambda$ \\
\textbf{context}, which is the tuple: \\
$\langle ingress\_port, \ egress\_port, \ physical\_port, \ key, \ meta, \ header\_offset, \ table \rangle$ \\ \\
Compactly, $\sigma = \langle \pi, \alpha, \Lambda, \textbf{\textrm{context}} \rangle$
\\ \\
We can examine or change the status of $\pi$ or \textbf{context} by using the index operator, []. The index itself is context-sensitive: $\pi$ is indexed by some \textit{field}, \textbf{context} is indexed by one of its substructures.
A field in $\pi$ can be set to a literal value or copied from a bitfield or register. For example:
\begin{equation}
  \pi[\textit{field}] \leftarrow \textit{value}
\end{equation}
\begin{equation}
  \pi[\textit{field}] \leftarrow \mathbb{r}
\end{equation}
\begin{equation}
  \pi[\textit{field}] \leftarrow \textit{bitfield}
\end{equation}

Comments:
\begin{enumerate}
\item Field \textit{field} with a packet, $\pi$, is set to some literal value, \textit{value}.
\item The value of register $\mathbb{r}$ is copied into \textit{field}.
\item The value of some bitfield, \textit{bitfield} is copied into \textit{field}.
\end{enumerate}


\subsection{Operational Semantics 5-Tuple}
% $ pip = \langle IF, \ CF, \ \Rightarrow, \ FC, \ OF \rangle $\\
% IF = $program \times \pi \rightarrow CF$ \\
% CF = $\alpha \times \xi \times \pi$ \\
% $\Rightarrow$ = Defined in section ``Reduction Relation''. \\
% FC = $(\Lambda = \emptyset) \times \pi \times \textbf{\textrm{context}}\langle egress\_port \neq 0 \rangle $ \\
% OF = $FC \rightarrow \textrm{serialization}$

\subsection{Reduction Relation}
\setlength{\mathindent}{0pt}

\begin{mdframed}
\begin{gather*}
  (\mathct{advance}~n \cdot \bar\alpha), \langle \pi, \Lambda, \Gamma \rangle
  \Rightarrow
  (\bar\alpha), \langle \pi, \Lambda, \Gamma[\mathct{header}
  \leftarrow \mathct{header} + n] \rangle
\\
  (\mathct{clear} \cdot \bar\alpha), \langle \pi, \Lambda, \Gamma \rangle
  \Rightarrow
  (\bar\alpha), \langle \pi, \varnothing, \Gamma \rangle
\\
  (\mathct{drop} \cdot \bar\alpha), \langle \pi, \Lambda, \Gamma \rangle
  \Rightarrow
  \varnothing, \langle \pi, \varnothing, \Gamma [\mathct{out} \leftarrow 0] \rangle
\\
  (\mathct{write} \ \textit{action} \cdot \bar\alpha), \langle \pi, \Lambda,
  \Gamma \rangle
  \Rightarrow
  (\bar\alpha), \langle \pi, \Lambda \cdot \textit{action}, \Gamma \rangle
\\
  (\mathct{goto} \ T \cdot \bar\alpha),
  \langle \pi, \Lambda, \Gamma \rangle
  \Rightarrow
  (\bar\alpha), \langle \pi, \Lambda,
  \Gamma[\mathct{current\_table} \leftarrow T] \rangle
\\
  (\mathct{set} \ \textit{field} \ \textit{value} \cdot \bar\alpha), \langle \pi,
  \Lambda, \Gamma \rangle
  \Rightarrow
  (\bar\alpha), \langle \pi', \Lambda, \Gamma' \rangle
\\
  (\mathct{output} \ \mathct{port} \ n \cdot \bar \alpha), \langle \pi, \Lambda,
  \Gamma \rangle
  \Rightarrow
  (\bar\alpha), \langle \pi, \Lambda, \Gamma[\mathct{egress\_port}
  \leftarrow \mathct{port} \ n] \rangle
\\
  (\mathct{copy} \ as_1 \ src \ as_2 \ dst \cdot \bar\alpha), \langle \pi,
  \Lambda, \Gamma \rangle
  \Rightarrow
  (\bar\alpha), \langle \pi', \Lambda, \Gamma' \rangle
\end{gather*}
\end{mdframed}

The $\mathct{advance}$ command advances the header offset by \textit{n} bits.
This can be used by a programmer to adjust the decoding offset relative to a
packet header.
The $\mathct{clear}$ command clears the stored action list. This can be used
by a program reset any accumulated actions if certain conditions are detected
during packet analysis (e.g., filtering flows based on UDP packet content).
The $\mathct{drop}$ command truncates both the action list and stored action list,
and sets the output port to 0 (the null port). 
The effect of this to cause the program to terminate such that the execution 
environment will not forward the packet.
 

% \begin{equation}
%   (\textrm{\textbf{clear}}) \textrm{,} \ \langle \pi, \ (\textbf{\textrm{clear}} \ \oplus \ \bar \alpha), \ \bar \Lambda, \ \textrm{\textbf{context}} \rangle
%   \Rightarrow \langle \pi, \ \bar \alpha, \ \emptyset, \  \textrm{\textbf{context}} \rangle
% \end{equation}

%% Remove all actions from stored action list, $\Lambda$.


%% \begin{equation}
%%   (\textrm{\textbf{write}} \ action), \ \langle \pi, \ (\textrm{\textbf{write}} \ \oplus \ \bar \alpha), \ \bar \Lambda,
%%   \textrm{\textbf{context}} \rangle \Rightarrow
%%   \langle \pi, \ \bar \alpha, \ \bar \Lambda \cdot action, \textrm{\textbf{context}} \rangle
%% \end{equation}

%% Append \textit{action} to stored action list, $\Lambda$.

%% \begin{equation}
%%   (\textrm{\textbf{goto}} \ \tau) \textrm{,} \ \langle \pi, \ (\textrm{\textbf{goto}} \ \oplus \ \bar \alpha), \ \bar \Lambda, \textrm{\textbf{context}}[table] \rangle
%%   \Rightarrow \langle \pi, \ \bar \alpha, \ \bar \Lambda, \textrm{\textbf{context}}[\tau] \rangle
%% \end{equation}

%% Change the \textbf{current\_table} variable in the context to $\tau$.

%% \begin{equation}
%% \begin{split}
%%   (\textrm{\textbf{set}} \ field \ value) \textrm{,} \ \langle \pi, (\textrm{\textbf{set}} \ \oplus \ \bar \alpha), \bar \Lambda,
%%   \textbf{\textrm{context}}[field] \rangle \\
%%   \Rightarrow \langle \pi, \bar \alpha, \bar \Lambda,
%%   \textbf{\textrm{context}}[field] = value \rangle
%% \end{split}
%% \end{equation}

%% Set the bitfield, \textit{field}, to the literal value, \textit{value}.

%% \begin{equation}
%% \begin{split}
%%   (\textrm{\textbf{output}} \ port) \textrm{,} \ \langle \pi, (\textrm{\textbf{output}} \ \oplus \ \bar \alpha), \bar \Lambda,
%%   \textbf{\textrm{context}}[egress\_port] \rangle \\
%%   \Rightarrow 
%%   \langle \pi, \bar \alpha, \bar \Lambda,
%%   \textbf{\textrm{context}}[egress\_port] = port \rangle
%% \end{split}
%% \end{equation}

%% Change the \textbf{egress\_port} variable in the context to \textit{port}, thus outputting the packet to \textit{port}.

%% \begin{equation}
%% \begin{split}
%%   (\textrm{\textbf{copy}} \ as_1 \ src \ as_2 \ dst \ n)
%%   \textrm{,} \ \langle \pi, (\textbf{\textrm{copy}} \ \oplus \ \bar \alpha), \bar \Lambda, \textrm{\textbf{context}} \rangle \\ \Rightarrow
%%   \langle \pi', \bar \alpha, \bar \Lambda, \textrm{\textbf{context}}' \rangle
%% \end{split}
%% \end{equation}

%% Copy n bits from bitfield \textit{src} of \textbf{address space} $as_1$ into bitfield \textit{dst} of \textbf{address space} $as_2$.

Definition of copy function: (move to displaymath)
\begin{equation}
(\textrm{\textbf{copy}} \ meta \ src \ packet \ dst \ n) \Rightarrow
  \langle \pi[dst] = \textrm{\textbf{context.meta}[dst, dst + n)} \rangle
\end{equation}
\begin{equation}
  (\textrm{\textbf{copy}} \ meta \ src \ header \ dst \ n) \Rightarrow
  \langle \pi[dst + header\_offset] =
          \textrm{\textbf{context.meta}[dst, dst + n)} \rangle
\end{equation}
\begin{equation}
  (\textrm{\textbf{copy}} \ meta \ src \ key \ dst \ n) \Rightarrow
  \langle \textbf{\textrm{context.key}}[dst] =
          \textrm{\textbf{context.meta}[dst, dst + n]} \rangle
\end{equation}
\begin{equation}
  (\textrm{\textbf{copy}} \ meta \ src \ key \ dst \ n) \Rightarrow
  \langle \textbf{\textrm{context}}[key] =
          [meta, meta + n] \rangle
\end{equation}
\begin{equation}
  (\textrm{\textbf{copy}} \ packet \ src \ packet \ dst \ n) \Rightarrow
  \langle \pi[dst] = \pi[src, src + n) \rangle
\end{equation}
\begin{equation}
  (\textrm{\textbf{copy}} \ packet \ src \ header \ dst \ n) \Rightarrow
  \langle \pi[dst + header\_offset] = \pi[src, src + n) \rangle
\end{equation}
\begin{equation}
  (\textrm{\textbf{copy}} \ packet \ src \ key \ dst \ n) \Rightarrow
  \langle \textrm{\textbf{context.key}[dst]} = \pi[src, src + n) \rangle
\end{equation}
\begin{equation}
  (\textrm{\textbf{copy}} \ header \ src \ header \ dst \ n) \Rightarrow
  \langle \pi[dst + header\_offset] = \pi[src + header\_offset, src + header\_offset + n) \rangle
\end{equation}
\begin{equation}
  (\textrm{\textbf{copy}} \ header \ src \ packet \ dst \ n) \Rightarrow
  \langle \pi[dst] = \pi[src + header\_offset, src + header\_offset + n) \rangle
\end{equation}
\begin{equation}
  (\textrm{\textbf{copy}} \ header \ src \ key \ dst \ n) \Rightarrow
  \langle \textrm{\textbf{context.key}[dst]} = \pi[src + header\_offset, src + header\_offset + n) \rangle
\end{equation}
\begin{equation}
  (\textrm{\textbf{copy}} \ header \ src \ meta \ dst \ n) \Rightarrow
  \langle \textrm{\textbf{context.meta}[dst]} = \pi[src + header\_offset, src + header\_offset + n) \rangle
\end{equation}
